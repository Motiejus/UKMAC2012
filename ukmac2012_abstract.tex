%% LyX 2.0.4 created this file.  For more info, see http://www.lyx.org/.
%% Do not edit unless you really know what you are doing.
\documentclass[english]{article}
\usepackage[T1]{fontenc}
\usepackage[utf8]{inputenc}
\usepackage{geometry}
\geometry{verbose,tmargin=2cm,bmargin=2cm,lmargin=2.5cm,rmargin=2.5cm}
\usepackage{babel}
\begin{document}

\title{Parallel Programming on the Tilera Manycore Platform: Comparing Gannet
and Erlang}


\author{Ashkan Tousimojarad, Motiejus Jak\v{s}tys, Wim Vanderbauwhede}


\date{School of Computing Science, University of Glasgow}

\maketitle
In this talk we report preliminary results of our work on novel parallel
programming approaches for manycore platforms. We introduce our novel
programming framework, Gannet, based on a functional task composition
language with parallel evaluation. The purpose of Gannet is to make
it possible for the programmer to express parallel algorithms with
complex interaction patterns (such as parallel pipelines and reductions)
in a very natural and easy way. We compare this approach to Erlang,
a well-known actor model based language.

As our aim is to demonstrate that our approach scales easily to large numbers
of cores, our target platform in this work is the Tilera TILE64 64-core
processor. We also show results on a conventional 2x~Quad-core multicore
processor Intel(R)~Xeon(R)~CPU~E5620 and high performance 2xIntel~Xeon~E5-2670,
eight-core processor, which can be rent on Amazon~EC2. We present results on
standard benchmarks {[}which benchmarks?{]} to illustrate the capability for
exploiting parallelism of each approach, as well as an implementation of the
merge sort algorithm.

The TILE64 processor is similar to a GPU in that it is an accelerator connect
to the PCIe bus. However, the system runs with a standard ANSI C and C++
programming environment, and is therefore not limited to data parallel
operation and does not require a special programming language such as CUDA or
OpenCL. TILE64 gives options to work like a standard Linux SMP machine (hence
most programs can run unchanged), or run on bare metal where a programmer sees
a 64-core network-on-chip without an operating system. Most useful are
intermediate options, such as Zero Overhead Linux, where operating system is
running in a way which does not cause any overhead.

\subsection*{More on Gannet}
{[}More on Gannet{]} 

\subsection*{More on Erlang}

We compare our Gannet approach to Erlang because Erlang was specifially
designed for distributed systems, so it share the aims of our work,
but is also a very mature platform.

Erlang is a programming language which has many features more commonly
associated with an operating system than with a programming language:
concurrent processes, scheduling, memory management, distribution, networking,
etc. 

Erlang is characterized by the following features:

\begin{description}
    \item[Concurrency] -- Erlang has extremely lightweight processes whose
        memory requirements can vary dynamically. Processes have no shared
        memory and communicate by asynchronous message passing. Erlang supports
        applications with very large numbers of concurrent processes. No
        requirements for concurrency are placed on the host operating system.
        However, once concurrency is available, it is very well exploited
        (reference needed).
    \item[Distribution] -- Erlang is designed to be run in a distributed
        environment. An Erlang virtual machine is called an Erlang node. A
        distributed Erlang system is a network of Erlang nodes (typically one
        per processor). An Erlang node can create parallel processes running on
        other nodes, which perhaps use other operating systems. Processes
        residing on different nodes communicate in exactly the same was as
        processes residing on the same node. 
    \item[Robustness] -- Erlang has various error detection primitives which can
        be used to structure fault-tolerant systems.
    \item[Soft real-time] -- Erlang supports programming "soft" real-time
        systems, which require response times in the order of milliseconds.
\end{description}

Also less important for HPC, but very important for enterprise applications:
\begin{itemize}
    \item Hot code upgrade
    \item Incremental code loading
\end{itemize}

Erlang was developed in Ericsson labs for its highly available phone switching
software. It is widely used in industry due to its reliability, scalability and
performance.

\subsection*{Remarks?}

Although this is very much a work in progress, we show that our approach
combines excellent performance with ease of use.

%The abstract is 1 page max, keep that in mind
\end{document}
