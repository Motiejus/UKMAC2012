%% LyX 2.0.4 created this file.  For more info, see http://www.lyx.org/.
%% Do not edit unless you really know what you are doing.
\documentclass[english]{article}
\usepackage[T1]{fontenc}
\usepackage[latin9]{inputenc}
\usepackage{geometry}
\geometry{verbose,tmargin=2cm,bmargin=2cm,lmargin=2.5cm,rmargin=2.5cm}
\usepackage{babel}
\begin{document}

\title{Parallel Programming on the Tilera Manycore Platform: Comparing Gannet
and Erlang}


\author{Ashkan Tousimojarad, Motiejus Jak\v{s}tys, Wim Vanderbauwhede }


\date{School of Computing Science, University of Glasgow}

\maketitle
In this talk we report preliminary results of our work on novel parallel
programming approaches for manycore platforms. We introduce our novel
programming framework, Gannet, based on a functional task composition
language with parallel evaluation. The purpose of Gannet is to make
it possible for the programmer to express parallel algorithms with
complex interaction patterns (such as parallel pipelines and reductions)
in a very natural and easy way. We compare this approach to Erlang,
a well-known actor model based language.

As our aim is to demonstrate that our approach scales easily to large
numbers of cores, our target platform in this work is the Tilera TILE64
64-core processor. However, we also show results on a conventional
multicore processor {[}list type here, I suggest you try it on roma{]}.
We present results on standard benchmarks to illustrate the capability
for exploiting parallelism of each approach, as well as an implementation
of the merge sort algorithm.

The TILE64 processor is similar to a GPU in that it is an accelerator
connect to the PCIe bus. However, the system runs Zero-Overhead Linux
and is therefore not limited to data parallel operation. It also does
not require a special programming language such as CUDA or OpenCL. 

{[}More on Gannet{]} 

We compare our Gannet approach to Erlang because Erlang was specifially
designed for distributed systems, so it share the aims of our work,
but is also a very mature platform.

{[}A bit more on Erlang{]}

Although this is very much a work in progress, we show that our approach
combines excellent performance with ease of use.

%The abstract is 1 page max, keep that in mind
\end{document}
